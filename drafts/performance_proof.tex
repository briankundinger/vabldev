\documentclass[12pt,letterpaper]{article}

\usepackage[hypertexnames=false]{hyperref}
\usepackage{amsmath}
\usepackage{graphicx}
\usepackage{enumerate}
\usepackage{natbib}

% Custom packages
\usepackage{amssymb}
\usepackage{booktabs}       % professional-quality tables
\usepackage{algorithm}      % algorithm environment
\usepackage{algpseudocode}
\usepackage{multirow}
\usepackage{sectsty}
\usepackage{tabularx}
\usepackage{tikz}           % vector graphics
\usepackage{bm}             % bold math symbols
\usepackage{xcolor}
\usepackage{microtype}
\usepackage{import}
\usepackage{titling}
\usepackage{natbib}
\usetikzlibrary{arrows, backgrounds, patterns, matrix, shapes, fit, 
  calc, shadows, plotmarks}


% Custom commands
\let\oldvec\vec
\renewcommand\vec{\bm}
\newcommand{\simfn}{\mathtt{sim}} % similarity function
\newcommand{\truncsimfn}{\underline{\simfn}} % truncated similarity function
\newcommand{\blockfn}{\mathtt{BlockFn}} % blocking function
\newcommand{\distfn}{\mathtt{dist}} % distance function
\newcommand{\valset}{\mathcal{V}} % attribute value set
\newcommand{\entset}{\mathcal{R}} % set of records that make up an entity
\newcommand{\partset}{\mathcal{E}} % set of entities that make up a partition
\newcommand{\1}[1]{\mathbb{I}\!\left[#1\right]} % indicator function
\newcommand{\euler}{\mathrm{e}} % Euler's constant
\newcommand{\dblink}{\texttt{\upshape \lowercase{d-blink}}} % Name of scalable Bayesian ER model
\newcommand{\blink}{\texttt{\upshape \lowercase{blink}}} % Name of original Bayesian ER model
\def\spacingset#1{\renewcommand{\baselinestretch}%
  {#1}\small\normalsize} \spacingset{1}

\newtheorem{remark}{Remark}
\newtheorem{proposition}{Proposition}
\newtheorem{definition}{Definition}
\newtheorem{lemma}{Lemma}

% DON'T change margins - should be 1 inch all around.
\addtolength{\oddsidemargin}{-.5in}%
\addtolength{\evensidemargin}{-.5in}%
\addtolength{\textwidth}{1in}%
\addtolength{\textheight}{1.3in}%
\addtolength{\topmargin}{-.8in}%

\sectionfont{\large\nohang\centering\MakeUppercase}

\title{Performance Metric Proofs for fabl/vabl}
\author{Brian Kundinger}

\begin{document}
\maketitle



\newpage
\spacingset{1.5}

\section{Initial Thoughts}

It seems to me that base Fellegi Sunter is bound to have a higher false positive rate as the size of the problem grows.

\begin{itemize}
	\item Assuming there are at maximum $K$ records in $A$ that match with record $j \in B$. For one to one matching, this means $K = 1$.
	\item There are at least $n_A - K$ nonmatching record pairs for each $j \in B$
	\item If we're making $n_A n_B$ independent classification decisions it just seems intuitive that we would get more false positives as $(n_A - K) n_B$ grows. 
	\item It also seems intuitive that we would tend to see less false positives when we can only make at most $n_B$ classification decisions. 
\end{itemize}
A possible approach: In FS a record pair gets classified as a match if it has an appropriately high $w_{ij}$. But in fabl, you need to have the high $w_{ij}$, and no other  $w_{i'j}$ can be greater. This might be tangible place to start?

Does that make sense? Does it seem possible to prove rigorously? Have you thought about this before?

If we could do this, it would give a nice theoretical justification for using the fabl framework over base FS (or practically, using vabl over fastLink). 

\section{An Attempt at Rigor}

Essentially, we would try to prove 
$$E[\text{\# False Positives} | \text{FS}] \geq E[\text{\# False Positives} | \text{fabl}].$$

We would assume all $m$ and $u$ parameters are the same for each model. We assume there are at most $K = 1$ matches in $X_1$ for each record in $X_2$. We can use the correspondence that $\lambda$ under base Fellegi Sunter is equal to $\frac{\pi}{n_A}$ under fabl. 

Left hand side:
\begin{align*}
	E[\text{\# False Positives} | \text{FS}] &= n_A n_B \times p(\text{False Positive}|\text{FS}) \\
	&=n_A n_B \times p\left(\frac{u_{ij}(1 - \lambda)}{m_{ij} \lambda  + u_{ij}(1 - \lambda)} \geq 0.5 \right) \\
\end{align*}

\bibliographystyle{jasa}
\bibliography{biblio}


\end{document}
